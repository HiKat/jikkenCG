\documentclass[a4j,dvipdfmx]{jsarticle}
%\usepackage{graphicx}
%\usepackage{multirow}
%\usepackage{color}
%\usepackage{lscape}
%\usepackage{ascmac}
%\usepackage{txfonts}


\usepackage{listings, jlisting}
\renewcommand{\lstlistingname}{リスト}
\lstset{
  language={C},
  basicstyle=\ttfamily\scriptsize,
  commentstyle=\textit,
  classoffset=1,
  keywordstyle=\bfseries,
  frame=tRBl,
  framesep=5pt,
  showstringspaces=true,
  numbers=left,
  stepnumber=1,
  numberstyle=\tiny,
  tabsize=2
}


%% %subsubsubsectionの定義(できるだけ使わない方向で)
%% \makeatletter
%% \newcommand{\subsubsubsection}{\@startsection{paragraph}{4}{\z@}%
%%   {1.0\Cvs \@plus.5\Cdp \@minus.2\Cdp}%
%%   {.1\Cvs \@plus.3\Cdp}%
%%   {\reset@font\sffamily\normalsize}
%% }
\makeatother
\setcounter{secnumdepth}{4}
%ここまでsubsubsubsectionの定義

\begin{document}

\title{計算機科学実験及び演習4\\コンピュータグラフィックス\\課題1}
\author{工学部情報学科3回生 1029255242\\勝見久央}
\date{作成日: \today} % コンパイル時の日付が自動で挿入される
\maketitle
%文字コードはUTF-8推奨.それ以外ではline2のcontentsline~にエラー発生.
%選択範囲コメントアウトは選択中にM-;
%%%%%%%%%%%%%%%%%%%%%%%%%%%%%%%%%%%%%%%%%%%%%%%%%%%%%%%%%%%%%%%%%%%%%%
%ソースコードの貼付け
%% \lstinputlisting[caption=キャプション,label=ラベル,breaklines=true]
%% {./kadai01.sc}
%%次でも可
%% \begin{lstlisting}
%% \end{lstlisting}
\section{概要}
本実験課題では3Dポリゴンデータを透視投影によって投影したPPM画像を生成する
プログラムをC言語で作成した.

\section{要求仕様}
作成したプログラムが満たす仕様は以下の通りである.
\begin{itemize}
\item ポリゴンデータはプログラム内部で与え、頂点座標をランダムに生成する.
\item PPM画像の大きさは$$256 \times 256$$
\item カメラ位置は(x,y,z) = (0.0, 0.0, 0.0)
\item カメラ方向あh(x,y,z) = (0.0, 0.0, 1.0)
\item カメラ焦点距離は256.0
\item ポリゴンには拡散反射を施す
\end{itemize}

\section{プログラムの仕様}
\subsection{留意点}
座標、RGB値等はデータ型をdoubleとして計算し、RGB値については出力時には
一旦round関数を用いて丸めてint型に変換した後char型に変換して出力した.
各点の座標についてはdouble型のサイズ3の配列にxyz座標を格納して処理した.
プログラム内には頂点座標の例として2パターン分もコメントで記述している.
大文字アルファベットの定数はマクロである.


\subsection{各種定数}
%======================================================================
\subsubsection{ppm}
次の定数はppmファイル生成のための定数である.
\begin{itemize}
\item FILENAME\\
  ファイル名を指定. ここではimage.ppmaとしている.
  
\item MAGICNUM\\
  ppmファルのヘッダに記述する識別子. P3を使用.
  
\item WIDTH, HEIGHT, WIDTH\_STRING, HEIGHT\_STRING\\
  出力画像の幅、高さ. ともに256とする. STRINGは文字列として使用するためのマクロ.
  以降も同様.
  
\item MAX, MAX\_STRING\\
  RGBの最大値. 255を使用.

\end{itemize}

%======================================================================
\subsubsection{ポリゴンデータ}
次の定数はポリゴンデータとして定める定数である.
\begin{itemize}
\item VER\_NUM\\
  ポリゴンの頂点数
  
\item SUR\_NUM\\
  ポリゴンを生成する三角形平面の数
  
\item ver[VER\_NUM][3]\\
  VRMLのCoordinateノードのpointフィールドの値.
  ポリゴンを形成する各点の座標を格納する.
  点iの座標は(sur[i][0], sur[i][1], sur[i][2])
  である. 初期化はmain関数内で行う.
  double2次元型配列.
  
\item sur[SUR\_NUM][3]\\
  VRMLのIndexedFaceSetノード内のcoordIndexフィールドの値.
  ポリゴンを形成する三角形を指定する.
  三角形は点sur[i][0]、sur[i][1]、sru[i][2]の3点
  からなる.int型2次元配列.
  
\item diffuse\_color[3]\\
  VRMLのMaterialノードのdiffuseColorフィールドの値.
  配列の先頭から順にRGB値を表す.
  double型配列.
\end{itemize}


%======================================================================
\subsubsection{環境設定}
次の定数は光源モデルなどの外部環境を特定する定数である.
\begin{itemize}
\item FOCUS\\
  カメラの焦点距離. 256.0を使用
  
\item light\_dir[3]\\
  光源方向ベクトル.doubel型配列
  
\item light\_rgb[3]\\
  光源の明るさを正規化したRGB値にして配列に格納したもの.
  double型配列.
  
\end{itemize}

%======================================================================
\subsubsection{その他}
\begin{itemize}
\item image[HEIGHT][WIDTH]\\
  描画した画像の各点の画素値を格納するための領域.
  領域確保のみで初期化は関数内で行う.
  double型の3次元配列.
  
\item projected\_ver[VER\_NUM][2]\\
  画像平面上に投影された各点の座標を格納するための領域.
  領域確保のみで初期化は関数内で行う.
  また最初にmain関数内で全ての点が黒になるように
  初期化を行ってからシェーディングの結果をその上に反映させていく形をとる.
  double型2次元配列.
\end{itemize}

%======================================================================

\subsection{関数仕様}
\begin{itemize}
\item double func1(double \*p, double \*q, double y)\\
  double型2次元配列で表された2点p、qの座標とdouble型の値yを引数に取り、
  直線pqと直線y=yの交点のx座標をdouble型で返す関数.
  ラスタライズの計算を簡素化するために三角形を分割する際に用いる.

\item int lineOrNot(double \*a, double \*b, double \*c)\\
  double型2次元配列で表された3点a、b、cが一直線上にあるかどうかを判別する関数.
  一直線上にある場合はint型1を返し、それ以外のときはint型0を返す.
  後述の関数shadingの中で用いる.

\item void perspective\_pro()\\
  大域変数で与えられた頂点座標ver[VER\_NUM][3]の各点に対して透視投影(pespective project)
  を行い、その結果をグローバル領域に確保された領域projected\_ver[VER\_NUM][2]に書き込む.

\item void shading(double *a, double *b, double *c, double *n)
  画像平面上に投影されたduble型2次元配列で与えられた3点a、b、cに対してシェーディングを行う関数.
  さらにシェーディング時に拡散反射をコンスタントシェーディングで適用するために必要になる3点a、b、c
  が座標空間内で形成するポリゴンの面の法線ベクトルをdouble型3次元配列の形にしたnを引数にとる.
  シェーディングの結果はグローバル領域にあるimage[HEIGHT][WIDTH]に書き込む.
   
\end{itemize}
\section{プログラム本体}
プログラム本体は次のようになった.
\lstinputlisting[caption=キャプション,label=ラベル,breaklines=true]{./kadai01-reflex.c}

\section{実行例}


\end{document}
